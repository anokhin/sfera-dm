\documentclass[10pt,a4paper]{article}
\usepackage[utf8]{inputenc}
\usepackage[russian]{babel}
\usepackage[OT1]{fontenc}
\usepackage{amsmath}
\usepackage{amsfonts}
\usepackage{amssymb}
\usepackage{graphicx}
\usepackage[left=1cm,right=1cm,top=0.5cm,bottom=0.5cm]{geometry}

\author{Nikolay Anokhin}

\begin{document}

\thispagestyle{empty}

\noindent Линейная регрессия \\
а) чувствительна к шуму \\
б) требует отсутствия полной коллинеарности факторов \\
в) метод наименьших квадратов позволяет получить достаточно качественные оценки параметров модели \\
г) не страдает от переобучения вследствие простоты модели \\

\noindent Логистическая регрессия \\
а) это статистическая модель, используемая для предсказания вероятности возникновения некоторого события путём подгонки данных к логистической кривой \\
б) для расчета применяют метод Ньютона \\
в) для расчета применяют стохастический градиентный спуск \\
г) применяется для решения задач классификации \\

\vspace{1em}
\noindent Линейная регрессия \\
а) чувствительна к шуму \\
б) требует отсутствия полной коллинеарности факторов \\
в) метод наименьших квадратов позволяет получить достаточно качественные оценки параметров модели \\
г) не страдает от переобучения вследствие простоты модели \\

\noindent Логистическая регрессия \\
а) это статистическая модель, используемая для предсказания вероятности возникновения некоторого события путём подгонки данных к логистической кривой \\
б) для расчета применяют метод Ньютона \\
в) для расчета применяют стохастический градиентный спуск \\
г) применяется для решения задач классификации \\

\vspace{1em}
\noindent Линейная регрессия \\
а) чувствительна к шуму \\
б) требует отсутствия полной коллинеарности факторов \\
в) метод наименьших квадратов позволяет получить достаточно качественные оценки параметров модели \\
г) не страдает от переобучения вследствие простоты модели \\

\noindent Логистическая регрессия \\
а) это статистическая модель, используемая для предсказания вероятности возникновения некоторого события путём подгонки данных к логистической кривой \\
б) для расчета применяют метод Ньютона \\
в) для расчета применяют стохастический градиентный спуск \\
г) применяется для решения задач классификации \\

\vspace{1em}
\noindent Линейная регрессия \\
а) чувствительна к шуму \\
б) требует отсутствия полной коллинеарности факторов \\
в) метод наименьших квадратов позволяет получить достаточно качественные оценки параметров модели \\
г) не страдает от переобучения вследствие простоты модели \\

\noindent Логистическая регрессия \\
а) это статистическая модель, используемая для предсказания вероятности возникновения некоторого события путём подгонки данных к логистической кривой \\
б) для расчета применяют метод Ньютона \\
в) для расчета применяют стохастический градиентный спуск \\
г) применяется для решения задач классификации \\

\vspace{1em}
\noindent Линейная регрессия \\
а) чувствительна к шуму \\
б) требует отсутствия полной коллинеарности факторов \\
в) метод наименьших квадратов позволяет получить достаточно качественные оценки параметров модели \\
г) не страдает от переобучения вследствие простоты модели \\

\noindent Логистическая регрессия \\
а) это статистическая модель, используемая для предсказания вероятности возникновения некоторого события путём подгонки данных к логистической кривой \\
б) для расчета применяют метод Ньютона \\
в) для расчета применяют стохастический градиентный спуск \\
г) применяется для решения задач классификации 

\end{document}