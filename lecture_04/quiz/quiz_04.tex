\documentclass[10pt,a4paper]{article}
\usepackage[utf8]{inputenc}
\usepackage[russian]{babel}
\usepackage[OT1]{fontenc}
\usepackage{amsmath}
\usepackage{amsfonts}
\usepackage{amssymb}
\usepackage{graphicx}
\usepackage[left=2cm,right=2cm,top=2cm,bottom=2cm]{geometry}

\author{Nikolay Anokhin}

\begin{document}

\thispagestyle{empty}

\section*{Тест по материалам занятия 3}

\subsection*{задание 1 (0.33)}

Агломеративная иерархическая кластеризация \\
а) на каждой итерации объединяет два ближайших кластера в один \\
б) разделяет кластер на кластеры меньшего размера \\
в) принимает решение на каждом шаге на основании одного из признаков \\
г) работает за $O(n \log n)$, где $n$ -- количество объектов

\subsection*{задание 2 (0.34)}

DBScan \\
а) Успешно работает при разных плотностях кластеров \\
б) Определяет количетсво кластеров автоматически \\
в) Хорошая реализация работает за $O(n \log n)$ \\
г) Хорошая реализация работает за $O(n)$

\subsection*{задание 3 (0.33)}

Adjusted Rand Index \\
a) Равен единице, когда кластеризации идентичны \\
б) Равен нулю, когда кластеризации идентичны \\
в) Равен количеству пар, попавших в один кластер в разных кластеризациях \\
г) Равен количеству пар, попавших в разные кластера в разных кластеризациях

\vspace{5em}

\section*{Тест по материалам занятия 3}

\subsection*{задание 1 (0.33)}

Агломеративная иерархическая кластеризация \\
а) на каждой итерации объединяет два ближайших кластера в один \\
б) разделяет кластер на кластеры меньшего размера \\
в) принимает решение на каждом шаге на основании одного из признаков \\
г) работает за $O(n \log n)$, где $n$ -- количество объектов

\subsection*{задание 2 (0.34)}

DBScan \\
а) Успешно работает при разных плотностях кластеров \\
б) Определяет количетсво кластеров автоматически \\
в) Хорошая реализация работает за $O(n \log n)$ \\
г) Хорошая реализация работает за $O(n)$

\subsection*{задание 3 (0.33)}

Adjusted Rand Index \\
a) Равен единице, когда кластеризации идентичны \\
б) Равен нулю, когда кластеризации идентичны \\
в) Равен количеству пар, попавших в один кластер в разных кластеризациях \\
г) Равен количеству пар, попавших в разные кластера в разных кластеризациях

\end{document}