\documentclass[10pt,a4paper]{article}
\usepackage[utf8]{inputenc}
\usepackage[russian]{babel}
\usepackage[OT1]{fontenc}
\usepackage{amsmath}
\usepackage{amsfonts}
\usepackage{amssymb}
\usepackage{graphicx}
\usepackage[left=2cm,right=2cm,top=2cm,bottom=2cm]{geometry}

\author{Nikolay Anokhin}

\begin{document}

\section*{Тест по материалам занятия 1}

\subsection*{задание 1 (0.5)}

Алгоритм обучения модели: \\
(а) возвращает значение вектора параметров модели \\
(б) определяет форму зависимости между признаками и целевой переменной \\
(в) принимает на вход обучающую выборку \\
(г) возвращает вектор вероятности принадлежности объекта к классу

\subsection*{задание 2 (0.5)}

В задаче предсказания погоды даны влажность, температура и давление воздуха, измеряемые каждый день в течении одного года. Также для каждого из дней дано значение целевой функции -- шел в этот день дождь или нет. При решении предполагается использование всех имеющихся в наличии данных. \\
(а) сформулированная задача является задачей обучения с учителем \\
(б) сформулированная задача является задачей обучения без учителя

\vspace{3em}

\section*{Тест по материалам занятия 1}

\subsection*{задание 1 (0.5)}

Алгоритм обучения модели: \\
(а) возвращает значение вектора параметров модели \\
(б) определяет форму зависимости между признаками и целевой переменной \\
(в) принимает на вход обучающую выборку \\
(г) возвращает вектор вероятности принадлежности объекта к классу

\subsection*{задание 2 (0.5)}

В задаче предсказания погоды даны влажность, температура и давление воздуха, измеряемые каждый день в течении одного года. Также для каждого из дней дано значение целевой функции -- шел в этот день дождь или нет. При решении предполагается использование всех имеющихся в наличии данных. \\
(а) сформулированная задача является задачей обучения с учителем \\
(б) сформулированная задача является задачей обучения без учителя

\vspace{3em}

\section*{Тест по материалам занятия 1}

\subsection*{задание 1 (0.5)}

Алгоритм обучения модели: \\
(а) возвращает значение вектора параметров модели \\
(б) определяет форму зависимости между признаками и целевой переменной \\
(в) принимает на вход обучающую выборку \\
(г) возвращает вектор вероятности принадлежности объекта к классу

\subsection*{задание 2 (0.5)}

В задаче предсказания погоды даны влажность, температура и давление воздуха, измеряемые каждый день в течении одного года. Также для каждого из дней дано значение целевой функции -- шел в этот день дождь или нет. При решении предполагается использование всех имеющихся в наличии данных. \\
(а) сформулированная задача является задачей обучения с учителем \\
(б) сформулированная задача является задачей обучения без учителя

\end{document}