\documentclass[10pt,a4paper]{article}
\usepackage[utf8]{inputenc}
\usepackage[russian]{babel}
\usepackage[OT1]{fontenc}
\usepackage{amsmath}
\usepackage{amsfonts}
\usepackage{amssymb}
\usepackage{graphicx}
\usepackage[left=1cm,right=1cm,top=0.5cm,bottom=0.5cm]{geometry}

\author{Nikolay Anokhin}

\begin{document}

\thispagestyle{empty}

\noindent Градиентный спуск \\
а) находит глобальный минимум \\
б) движение к минимуму на каждом шаге происходит в направлении вектора градиента \\
в) в зависимости от способа выбора шага может расходиться \\
г) можно модифицировать для обучения ``онлайн''

\vspace{1em}
\noindent SVM \\
а) не чувствителен к выбросам \\
б) находит глобальный минимум \\
в) подбирает функцию ядра автоматически \\
г) подходит для работы с огромными наборами данных

\vspace{2em}
\noindent Градиентный спуск \\
а) находит глобальный минимум \\
б) движение к минимуму на каждом шаге происходит в направлении вектора градиента \\
в) в зависимости от способа выбора шага может расходиться \\
г) можно модифицировать для обучения ``онлайн''

\vspace{1em}
\noindent SVM \\
а) не чувствителен к выбросам \\
б) находит глобальный минимум \\
в) подбирает функцию ядра автоматически \\
г) подходит для работы с огромными наборами данных

\vspace{2em}
\noindent Градиентный спуск \\
а) находит глобальный минимум \\
б) движение к минимуму на каждом шаге происходит в направлении вектора градиента \\
в) в зависимости от способа выбора шага может расходиться \\
г) можно модифицировать для обучения ``онлайн''

\vspace{1em}
\noindent SVM \\
а) не чувствителен к выбросам \\
б) находит глобальный минимум \\
в) подбирает функцию ядра автоматически \\
г) подходит для работы с огромными наборами данных

\vspace{2em}
\noindent Градиентный спуск \\
а) находит глобальный минимум \\
б) движение к минимуму на каждом шаге происходит в направлении вектора градиента \\
в) в зависимости от способа выбора шага может расходиться \\
г) можно модифицировать для обучения ``онлайн''

\vspace{1em}
\noindent SVM \\
а) не чувствителен к выбросам \\
б) находит глобальный минимум \\
в) подбирает функцию ядра автоматически \\
г) подходит для работы с огромными наборами данных

\vspace{2em}
\noindent Градиентный спуск \\
а) находит глобальный минимум \\
б) движение к минимуму на каждом шаге происходит в направлении вектора градиента \\
в) в зависимости от способа выбора шага может расходиться \\
г) можно модифицировать для обучения ``онлайн''

\vspace{1em}
\noindent SVM \\
а) не чувствителен к выбросам \\
б) находит глобальный минимум \\
в) подбирает функцию ядра автоматически \\
г) подходит для работы с огромными наборами данных

\end{document}