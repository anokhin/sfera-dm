\documentclass[11pt]{article}

\usepackage[utf8]{inputenc}
\usepackage[russian]{babel}
\usepackage[OT1]{fontenc}
\usepackage{amsmath}
\usepackage{amsfonts}
\usepackage{amssymb}
\usepackage{graphicx}
\usepackage[left=2cm,right=2cm,top=2cm,bottom=2cm]{geometry}

\renewcommand{\labelitemi}{$\Box$}
\renewcommand{\labelitemii}{$\Box$}
\renewcommand{\labelitemiii}{$\Box$}
\renewcommand{\labelitemiv}{$\Box$}

\setlength{\parskip}{1em} % 1ex plus 0.5ex minus 0.2ex}
\setlength{\parindent}{0pt}

\pagenumbering{gobble}

\author{Николай Анохин}

\begin{document}

\small

%\section*{Вариант 1}
%
%1. Найдите собственные числа матрицы
%\[
%A = \begin{pmatrix}
%0 & -2 \\
%3 & 5
%\end{pmatrix}
%\]
%\begin{itemize}
%\item $\lambda_1 = 4,\;\lambda_2=3$ 
%\item $\lambda_1 = 3,\;\lambda_2=2$
%\item $\lambda_1 = -2,\;\lambda_2=-3$
%\item $\lambda_1 = 3,\;\lambda_2=3$
%\end{itemize}
%
%2. Чему равен косинус угла между векторами $u$ и $v$?
%\[
%u = \begin{pmatrix}
%3 \\
%4 
%\end{pmatrix}, \;\;
%v = \begin{pmatrix}
%4 \\
%3
%\end{pmatrix}
%\]
%\begin{itemize}
%\item $24/25$ 
%\item $3/4$
%\item $9/16$
%\item $3\pi/4$
%\end{itemize}
%
%3. Чему равно математическое ожидание суммы выпавших очков после трех бросков честного игрального кубика?
%\begin{itemize}
%\item $3.5$ 
%\item $10.5$
%\item $14.0$
%\item $7.0$
%\end{itemize}
%
%4. В хеш-таблице из $m$ ячеек содержится $n \gg m$ элементов. Хранение элементов внутри ячейки организовано с помощью двусвязного списка. При условии хорошей хеш-функции, какова алгоритмическая сложность поиска элемента в этой хеш-таблице?
%\begin{itemize}
%\item $O(m)$ 
%\item $O(1)$
%\item $O(n/m)$
%\item $O(n)$
%\end{itemize}
%
%5. В пространстве неотрицательных целых чисел, является ли функция $d(x, y) = max(x,y)$ корректно определенной функцией расстояния?
%\begin{itemize}
%\item Да, является
%\item Нет, не является
%\end{itemize}
%
%\newpage

\section*{В1. Чаепитие на работе}

Два друга после окончания института устроились на работу в крупную интернет-компанию. Офис компании располагается в новом 26-этажном здании, и все сотрудники хотят иметь рабочие места с красивым видом на одном из верхних этажей. Чтобы никому не было обидно, ежегодно проводится честная лотерея: каждый сотрудник распределяется на случайно (равномерно) выбранный этаж. Друзья договорились, что будут как можно чаще навещать друг друга и вместе пить чай. При предположении, что они проработают в компании много лет, сколько в среднем пролетов между этажами будет разделять друзей?

\section*{В2. Заслуженный отдых}

После выхода на пенсию два программиста, по многу лет работавшие в крупной интернет-компании, решили построить себе по загородному дому. Каждый из них случайным образом выбрал себе участок на средиземноморском побережье между Ниццей и Сан-Ремо. Какова вероятность, что они смогут пешком ходить друг к другу в гости, если расстояние между Ниццей и Сан-Ремо примерно 60 км, а пожилой программист не способен пройти больше 10 км в одну сторону?

\section*{В3. Рабочие места}

Офис крупной интернет-компании располагается в новом 26-этажном здании, поэтому все сотрудники хотят иметь рабочие места с красивым видом на одном из верхних этажей. Чтобы никому не было обидно, ежегодно проводится честная лотерея: каждый сотрудник распределяется на случайно (равномерно) выбранный этаж. На каждом этаже всего 100 рабочих мест. При предположении, что в компании 2600 сотрудников, чему равно математическое ожидание количества сотрудников, которые попадут на этаж, где все места уже заняты? {\it Примечание: такая ситуация может возникнуть, если на какой-то этаж было распределено больше 100 человек -- соответственно на других этажах остались свободные места.}

%\newpage
%
%\section*{Вариант 2}
%
%1. Найдите собственные числа матрицы
%\[
%A = \begin{pmatrix}
%0 & -2 \\
%2 & 4
%\end{pmatrix}
%\]
%\begin{itemize}
%\item $\lambda_1 = 4,\;\lambda_2=1$ 
%\item $\lambda_1 = 2,\;\lambda_2=-2$
%\item $\lambda_1 = 1,\;\lambda_2=-1$
%\item $\lambda_1 = 2,\;\lambda_2=2$
%\end{itemize}
%
%2. Чему равен косинус угла между векторами $u$ и $v$?
%\[
%u = \begin{pmatrix}
%-3 \\
%-4 
%\end{pmatrix}, \;\;
%v = \begin{pmatrix}
%4 \\
%3
%\end{pmatrix}
%\]
%\begin{itemize}
%\item $-3\pi/4$
%\item $-24/25$
%\item $9/16$
%\item $-24/25$
%\end{itemize}
%
%3. Чему равно математическое ожидание суммы выпавших очков после двух бросков честного игрального кубика?
%\begin{itemize}
%\item $3.5$ 
%\item $10.5$
%\item $14.0$
%\item $7.0$
%\end{itemize}
%
%4. В хеш-таблице из $m$ ячеек содержится $n \gg m$ элементов. Хранение элементов внутри ячейки организовано с помощью двусвязного списка. При условии хорошей хеш-функции, какова алгоритмическая сложность вставки элемента в эту хеш-таблицу?
%\begin{itemize}
%\item $O(m)$ 
%\item $O(1)$
%\item $O(n/m)$
%\item $O(n)$
%\end{itemize}
%
%5. В пространстве неотрицательных целых чисел, является ли функция $d(x, y) = | x - y |$ корректно определенной функцией расстояния?
%\begin{itemize}
%\item Да, является
%\item Нет, не является
%\end{itemize}
%
%\newpage
%
%\section*{Генералы на пенсии}
%
%\newpage
%
%\section*{Вариант 3}
%
%1. Найдите собственные числа матрицы
%\[
%A = \begin{pmatrix}
%0 & -3 \\
%2 & 7
%\end{pmatrix}
%\]
%\begin{itemize}
%\item $\lambda_1 = 6,\;\lambda_2=1$ 
%\item $\lambda_1 = 7,\;\lambda_2=-1$
%\item $\lambda_1 = 2,\;\lambda_2=-3$
%\item $\lambda_1 = -6,\;\lambda_2=-1$
%\end{itemize}
%
%2. Чему равен косинус угла между векторами $u$ и $v$?
%\[
%u = \begin{pmatrix}
%-3 \\
%4 
%\end{pmatrix}, \;\;
%v = \begin{pmatrix}
%-4 \\
%3
%\end{pmatrix}
%\]
%\begin{itemize}
%\item $-24/25$
%\item $24/25$
%\item $9/16$
%\item $-3\pi/4$
%\end{itemize}
%
%3. Чему равно математическое ожидание суммы выпавших очков после четырех бросков честного игрального кубика?
%\begin{itemize}
%\item $3.5$ 
%\item $10.5$
%\item $14.0$
%\item $7.0$
%\end{itemize}
%
%4. В хеш-таблице из $m$ ячеек содержится $n \gg m$ элементов. Хранение элементов внутри ячейки организовано с помощью двусвязного списка. При условии хорошей хеш-функции, какова алгоритмическая сложность удаления элемента из этой хеш-таблицы?
%\begin{itemize}
%\item $O(m)$ 
%\item $O(1)$
%\item $O(n/m)$
%\item $O(n)$
%\end{itemize}
%
%5. В пространстве неотрицательных целых чисел, является ли функция $d(x, y) = x + y$ корректно определенной функцией расстояния?
%\begin{itemize}
%\item Да, является
%\item Нет, не является
%\end{itemize}
%
%\newpage
%
%\section*{Рассеяные влюблюенные}

\end{document}