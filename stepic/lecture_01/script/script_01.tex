\documentclass[10pt,a4paper]{article}

\usepackage[utf8]{inputenc}
\usepackage[russian]{babel}
\usepackage[OT1]{fontenc}
\usepackage{amsmath}
\usepackage{amsfonts}
\usepackage{amssymb}
\usepackage{graphicx}
\usepackage{hyperref}
\usepackage[left=2cm,right=2cm,top=2cm,bottom=2cm]{geometry}

\author{Nikolay Anokhin}

\begin{document}

\section{Введение}

%%%%%%%%%%%%%%%%%%%%%%%%%%%

\section{Data Mining как KDD}

\paragraph{Источники} \cite{journeys} \cite{kdd89} \cite{critic}

\begin{enumerate}
\item Определение, данное на слайде часто цитируется во многих источниках. Оно было предложено в статье Fayyad et.al в 1996 году, но сам термин KDD появился на 7 лет раньше.
\item KDD - название, предложенное Григорием Пятецким-Шапиро для семинара, который он организовывал в рамках конференции IJCAI-89. Про воркшоп поговорим чуть позже в рамках исторической справки.
\item Хотя это определение часто цитируется в литературе, оно имеет ряд существенных неточностей.
\item Если интерпретировать valid, как ``точный'', то зачастую возникает противоречие с остальными перечисленными качествами.
\item Расмотрим, например, задачу анализа переписи населеня. Одна из очень точных закономерностей, которую можно извлечь - женщины не служат в армии. Немотря на высокую точность, у этой закономерности нет ни новизны, ни практической полезности.
\item Есть также и проблемы с интерпретируемостью закономмерностей. В случае, когда решение принимается автоматической системой, она не нужна. А для человека интепретируемость - слишком субъективное понятие.
\item Учитывая перечисленные недостатки данного определения, делаем вывод, что нам понадобится что-то получше.
\end{enumerate}

%%%%%%%%%%%%%%%%%%%%%%%%%%%

\section{Data Mining как моделирование}

\paragraph{Источники} \cite{mmds}

%%%%%%%%%%%%%%%%%%%%%%%%%%%

\section{Некоторые важные события в истории Data Mining}

\paragraph{Источники} \cite{journeys}

%%%%%%%%%%%%%%%%%%%%%%%%%%%

\section{Data Mining -- дисциплина тысячи имен}

\paragraph{Источники} \cite{journeys}

\begin{enumerate}
\item 
\end{enumerate}

%%%%%%%%%%%%%%%%%%%%%%%%%%%%%%%%%%%%%%%%%%%%%%%%%%%%%%

\begin{thebibliography}{10} 

\bibitem{journeys} \href{http://citeseerx.ist.psu.edu/viewdoc/download?doi=10.1.1.363.1177&rep=rep1&type=pdf}{Journeys to Data Mining: The Journey of Knowledge Discovery}

\bibitem{kdd89} \href{http://www.kdnuggets.com/gpspubs/sigkdd-explorations-kdd-10-years.html}{Knowledge Discovery in Databases: 10 years after}

\bibitem{critic} \href{http://www.pantaneto.co.uk/issue30/Freitas.htm}{Are We Really Discovering ``Interesting'' Knowledge From Data?}

\bibitem{mmds} \href{http://infolab.stanford.edu/~ullman/mmds/book.pdf}{Mining of Massive Datasets}

\end{thebibliography}

\end{document}