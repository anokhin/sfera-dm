\documentclass[11pt]{beamer}
%\usetheme{Marburg}
\usepackage[utf8]{inputenc}
\usepackage[russian]{babel}
\usepackage[OT1]{fontenc}
\usepackage{amsmath}
\usepackage{amsfonts}
\usepackage{amssymb}
\usepackage{graphicx}

\author{Николай Анохин}

\title{Краткое введение в data mining}
%\setbeamercovered{transparent} 
\setbeamertemplate{navigation symbols}{} 
%\logo{} 
%\institute{} 
\date{} 
%\subject{}

\begin{document}

\begin{frame}
\titlepage
\end{frame}

\section{Задача data mining}

\begin{frame}{Data Mining как KDD}

\begin{quote}{Knowledge Discovery in Databases (KDD)}
-- это процесс получения точных, неизвестных, потенциально полезных и интерпретируемых закономерностей из данных.\footnote{U. Fayyad, G. Piatetsky-Shapiro, P. Smyth. From data mining to knowledge discovery: an overview. 1996}
\end{quote}

\end{frame}

\begin{frame}{Data Mining как моделирование}

\begin{quote}{Data Mining}
-- процесс построения модели, хорошо описывающей закономерности, которые порождают данные.
\end{quote}

Подходы к построению моделей
\begin{itemize}
\item cтатистический
\item машинное обучение
\item вычислительный
\end{itemize}

\end{frame}

\begin{frame}{Пример 1. Книга -- лучший подарок}

\begin{tabular}{l | c | c | c | c | c | c | c | c | c | c |}
PRML / C. Bishop \\
\hline
Ulysses / J. Joyce
\end{tabular}

\end{frame}

\begin{frame}{Некоторые важные события в истории Data Mining}

\begin{enumerate}
\item[1989] IJCAI-89 Workshop on Knowledge Discovery in Databases 
\end{enumerate}

\end{frame}

\begin{frame}{Data Mining -- дисциплина тысячи имен}

\begin{enumerate}
\item[1960-е] Data Fishing, Data Dredging
\item[1980-е] Knowledge Discovery in Databases
\item[1990-е] Data Mining, Database mining\textsuperscript{TM}
\item[2000-е] Data Analytics, Data Science\footnote{\href{https://twitter.com/nivertech/status/180109930139893761}{Data Scientist is a Data Analyst who lives in California}}\footnote{\href{https://twitter.com/josh_wills/status/198093512149958656}{A data scientist is someone who is better at statistics than any software engineer and better at software engineering than any statistician.}}
\end{enumerate}

\end{frame}


\end{document}