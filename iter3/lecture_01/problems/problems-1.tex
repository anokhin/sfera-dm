\documentclass[12pt,a4paper]{article}

\usepackage[utf8]{inputenc}
\usepackage[russian]{babel}
\usepackage[OT1]{fontenc}
\usepackage{amsmath}
\usepackage{amsfonts}
\usepackage{amssymb}
\usepackage{makeidx}
\usepackage{graphicx}
\usepackage[left=2cm,right=2cm,top=2cm,bottom=2cm]{geometry}

\author{Николай Анохин}

\begin{document}

\section*{Занятие 1}

\subsection*{Задача 1}

Пусть имеется информация о покупках, совершаемых 100 миллионами людей, каждый из которых ходит в магазин в среднем 100 раз в году и покупает 10 из 1000 предложенных товаров. Предположим, что два человека попадают под подозрение, если они купили в течение года в точности один и тот же набор товаров (возможно, для изготовления бомбы?). С помощью принципа Бонферрони определите, будет ли эффективным метод выявления террористов, основанный на поиске таких пар людей.

\end{document}