\documentclass[10pt,a4paper]{article}
\usepackage[utf8]{inputenc}
\usepackage[russian]{babel}
\usepackage[OT1]{fontenc}
\usepackage{amsmath}
\usepackage{amsfonts}
\usepackage{amssymb}
\usepackage{graphicx}
\usepackage[left=2cm,right=2cm,top=2cm,bottom=2cm]{geometry}

\author{Nikolay Anokhin}

\begin{document}

\thispagestyle{empty}

\section*{Тест по материалам занятия 5}

\noindent 1-Nearest Neighbour (0.33) \\
а) имеет низкий bias и высокий variance \\
б) имеет высокий bias и низкий variance \\
в) bias и variance примерно равны \\

\noindent метод наименьших квадратов с небольшим количеством весов (0.33) \\
а) имеет низкий bias и высокий variance \\
б) имеет высокий bias и низкий variance \\
в) bias и variance примерно равны \\

\noindent Переобучение (0.34) \\
а) возникает, когда модель слишком сильно подстраивается под обучающие данные \\
б) является негативным эффектом, с которым принято бороться \\
в) скорее всего возникнет, когда данных много, а параметров модели мало \\
г) можно наблюдать, применив модель на независимой (тестовой) выборке \\

\section*{Тест по материалам занятия 5}

\noindent 1-Nearest Neighbour (0.33) \\
а) имеет низкий bias и высокий variance \\
б) имеет высокий bias и низкий variance \\
в) bias и variance примерно равны \\

\noindent метод наименьших квадратов с небольшим количеством весов (0.33) \\
а) имеет низкий bias и высокий variance \\
б) имеет высокий bias и низкий variance \\
в) bias и variance примерно равны \\

\noindent Переобучение (0.34) \\
а) возникает, когда модель слишком сильно подстраивается под обучающие данные \\
б) является негативным эффектом, с которым принято бороться \\
в) скорее всего возникнет, когда данных много, а параметров модели мало \\
г) можно наблюдать, применив модель на независимой (тестовой) выборке \\

\section*{Тест по материалам занятия 5}

\noindent 1-Nearest Neighbour (0.33) \\
а) имеет низкий bias и высокий variance \\
б) имеет высокий bias и низкий variance \\
в) bias и variance примерно равны \\

\noindent метод наименьших квадратов с небольшим количеством весов (0.33) \\
а) имеет низкий bias и высокий variance \\
б) имеет высокий bias и низкий variance \\
в) bias и variance примерно равны \\

\noindent Переобучение (0.34) \\
а) возникает, когда модель слишком сильно подстраивается под обучающие данные \\
б) является негативным эффектом, с которым принято бороться \\
в) скорее всего возникнет, когда данных много, а параметров модели мало \\
г) можно наблюдать, применив модель на независимой (тестовой) выборке \\

\end{document}