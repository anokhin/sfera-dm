\documentclass[10pt,a4paper]{article}
\usepackage[utf8]{inputenc}
\usepackage[russian]{babel}
\usepackage[OT1]{fontenc}
\usepackage{amsmath}
\usepackage{amsfonts}
\usepackage{amssymb}
\usepackage{graphicx}
\usepackage[left=2cm,right=2cm,top=2cm,bottom=2cm]{geometry}

\author{Nikolay Anokhin}

\begin{document}

\thispagestyle{empty}

\section*{Тест по материалам занятия 4}

BIRCH (0.5) \\
а) работает за $O(n)$ \\
б) умеет определять выбросы \\
в) когда требуется присвоить объектам из выборки метку кластера, за первый проход изучает данные, за второй выполняет кластеризацию \\

\noindent t-SNE (0.5) \\
a) минимизирует схожесть между объектами в исходном пространстве \\
б) минимизирует схожесть между объектами в целевом пространстве \\
в) минимизирует другой критерий \\
г) повторный запуск алгоритма на тех же данных в общем случае приводит к другому результату \\

\section*{Тест по материалам занятия 4}

BIRCH (0.5) \\
а) работает за $O(n)$ \\
б) умеет определять выбросы \\
в) когда требуется присвоить объектам из выборки метку кластера, за первый проход изучает данные, за второй выполняет кластеризацию \\

\noindent t-SNE (0.5) \\
a) минимизирует схожесть между объектами в исходном пространстве \\
б) минимизирует схожесть между объектами в целевом пространстве \\
в) минимизирует другой критерий \\
г) повторный запуск алгоритма на тех же данных в общем случае приводит к другому результату \\

\section*{Тест по материалам занятия 4}

BIRCH (0.5) \\
а) работает за $O(n)$ \\
б) умеет определять выбросы \\
в) когда требуется присвоить объектам из выборки метку кластера, за первый проход изучает данные, за второй выполняет кластеризацию \\

\noindent t-SNE (0.5) \\
a) минимизирует схожесть между объектами в исходном пространстве \\
б) минимизирует схожесть между объектами в целевом пространстве \\
в) минимизирует другой критерий \\
г) повторный запуск алгоритма на тех же данных в общем случае приводит к другому результату \\

\section*{Тест по материалам занятия 4}

BIRCH (0.5) \\
а) работает за $O(n)$ \\
б) умеет определять выбросы \\
в) когда требуется присвоить объектам из выборки метку кластера, за первый проход изучает данные, за второй выполняет кластеризацию \\

\noindent t-SNE (0.5) \\
a) минимизирует схожесть между объектами в исходном пространстве \\
б) минимизирует схожесть между объектами в целевом пространстве \\
в) минимизирует другой критерий \\
г) повторный запуск алгоритма на тех же данных в общем случае приводит к другому результату \\

\end{document}