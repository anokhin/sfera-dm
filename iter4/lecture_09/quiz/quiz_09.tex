\documentclass[10pt,a4paper]{article}
\usepackage[utf8]{inputenc}
\usepackage[russian]{babel}
\usepackage[OT1]{fontenc}
\usepackage{amsmath}
\usepackage{amsfonts}
\usepackage{amssymb}
\usepackage{graphicx}
\usepackage[left=2cm,right=2cm,top=2cm,bottom=2cm]{geometry}

\author{Nikolay Anokhin}

\begin{document}

\thispagestyle{empty}

\section*{Тест по материалам 8 занятия}

\noindent 1. Сопоставьте типы impurity типам задач, которые можно решать с их помощью \\
1) Gini impurity \\
2) Entropy \\ 
3) Variance \\
\\
а) Classification \\
б) Regression \\
в) Оба \\

\noindent 2. Для контроля переобучения в дереве решений можно: \\
1) Ограничить глубину дерева \\
2) Ограничить количество элементов в листьях \\
3) Использовать surrogate split \\
4) Обрезать ветви после построения дерева \\
5) Заранее останавливать разделение узлов \\
6) Слуайным образом перемешивать обучающую выборку

\section*{Тест по материалам 8 занятия}

\noindent 1. Сопоставьте типы impurity типам задач, которые можно решать с их помощью \\
1) Gini impurity \\
2) Entropy \\ 
3) Variance \\
\\
а) Classification \\
б) Regression \\
в) Оба \\

\noindent 2. Для контроля переобучения в дереве решений можно: \\
1) Ограничить глубину дерева \\
2) Ограничить количество элементов в листьях \\
3) Использовать surrogate split \\
4) Обрезать ветви после построения дерева \\
5) Заранее останавливать разделение узлов \\
6) Слуайным образом перемешивать обучающую выборку

\section*{Тест по материалам 8 занятия}

\noindent 1. Сопоставьте типы impurity типам задач, которые можно решать с их помощью \\
1) Gini impurity \\
2) Entropy \\ 
3) Variance \\
\\
а) Classification \\
б) Regression \\
в) Оба \\

\noindent 2. Для контроля переобучения в дереве решений можно: \\
1) Ограничить глубину дерева \\
2) Ограничить количество элементов в листьях \\
3) Использовать surrogate split \\
4) Обрезать ветви после построения дерева \\
5) Заранее останавливать разделение узлов \\
6) Слуайным образом перемешивать обучающую выборку

\end{document}