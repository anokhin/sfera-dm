\documentclass[10pt]{beamer}

\usetheme{default}

\usepackage[utf8]{inputenc}
\usepackage[russian]{babel}
\usepackage[OT1]{fontenc}
\usepackage{amsmath}
\usepackage{amsfonts}
\usepackage{amssymb}
\usepackage{graphicx}
\usepackage{etoolbox}
\usepackage{caption}
\usepackage{subcaption}
\usepackage{pifont}
\usepackage{xcolor}
\usepackage{framed}
\definecolor{shadecolor}{cmyk}{0,0,0,1}

\makeatletter

\setbeamercolor{title}{fg=white}
\setbeamercolor{frametitle}{fg=black}
\setbeamerfont*{title}{family=\sffamily,size=\LARGE}

\setbeamerfont{page number in head/foot}{size=\scriptsize}
\setbeamertemplate{footline}[frame number]
\let\otp\titlepage
\renewcommand{\titlepage}{\otp\addtocounter{framenumber}{-1}}

\setbeamertemplate{background canvas}{%
	\ifnumequal{\c@framenumber}{0}{%
      \includegraphics[width=\paperwidth,height=\paperheight]{images/cover.png}
   }{%
      \ifnumequal{\c@framenumber}{\inserttotalframenumber}{
         \includegraphics[width=\paperwidth,height=\paperheight]{images/back.png}
      }{%
         % Other frames
      }%
   }%
}

\makeatother

\beamertemplatenavigationsymbolsempty

\author{Николай Анохин}
\title{\newline \newline \newline Лекция 1 \\ Задачи Data Mining}

\begin{document}

\begin{frame}[plain]
\titlepage
\end{frame}

\begin{frame}{План занятия}
\tableofcontents
\end{frame}

% =======================
\section{Задача кластеризации}
% =======================

\begin{frame}{Задача кластеризации}

В задачах кластеризации целевая переменная не задана. Цель -- отыскать ``скрытую структуру'' данных.

\vspace{1em}
Зачем вообще рассматривать задачи без целевой переменной?
\begin{enumerate}
\item разметка данных -- дорогое удовольствие
\item можно сначала поделить, а потом разметить
\item возможность отслеживать эволюционные изменения
\item построение признаков
\item exploratory data analysis
\end{enumerate}

\end{frame}

\begin{frame}{Пример 1}

\end{frame}

\begin{frame}{Пример 2}

\end{frame}

\begin{frame}{Топ 1000 самых посещаемых доменов рунета}

T-SNE + DBSCAN

\end{frame}

\begin{frame}{Постановка задачи}

{\bf Дано.} $N$ обучающих $D$-мерных объектов $\mathbf{x}_i \in \mathcal{X}$, образующих тренировочный набор данных (training data set) $X$.

\vspace{1em}
{\bf Найти.} Модель $h^*(\mathbf{x})$ из семейства параметрических функций $H = \{h(\mathbf{x, \mathbf{\theta}}): \mathcal{X} \times \Theta \rightarrow \mathbb{N}\}$, ставящую в соответствие произвольному $\mathbf{x} \in \mathcal{X}$ один из $K$ кластеров так, чтобы объекты внутри одного кластера были похожи, а объекты из разных кластеров различались.

\vspace{1em}
\begin{itemize}
\item Как определить похожесть объектов?
\item Как оценить качество модели?
\item Как выбрать $K$?
\end{itemize}

\end{frame}

% =======================
\section{Статистический подход: смесь распределений}
% =======================

\begin{frame}{Многомерное нормальное распределение}

\end{frame}

\begin{frame}[plain]
\begin{center}
{\Large Вопросы}
\end{center}
\end{frame}

\end{document}